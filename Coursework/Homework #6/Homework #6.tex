
%%%%%%%%%%%%%%%%%%%%%%%%%%%%%%%%%%%%%%%%%%%%%%%%%%%%%%%%%%%%%%%%%%%%%%%%%%%%%%%%%%%%%%%
%%%%%%%%%%%%%%%%%%%%%%%%%%%%%%%%%%%%%%%%%%%%%%%%%%%%%%%%%%%%%%%%%%%%%%%%%%%%%%%%%%%%%%%
% 
% This top part of the document is called the 'preamble'.  Modify it with caution!
%
% The real document starts below where it says 'The main document starts here'.

\documentclass[12pt]{article}

\usepackage{amssymb,amsmath,amsthm}
\usepackage[top=1in, bottom=1in, left=1.25in, right=1.25in]{geometry}
\usepackage{fancyhdr}
\usepackage{enumerate}
\usepackage{listings}
\usepackage{graphicx}
\usepackage{float}

\usepackage{mwe}
\usepackage{caption}
\usepackage{subcaption}
% Comment the following line to use TeX's default font of Computer Modern.
\usepackage{times,txfonts}



\makeatletter
\renewcommand*\env@matrix[1][*\c@MaxMatrixCols c]{%
  \hskip -\arraycolsep
  \let\@ifnextchar\new@ifnextchar
  \array{#1}}
\makeatother

\newtheoremstyle{homework}% name of the style to be used
  {18pt}% measure of space to leave above the theorem. E.g.: 3pt
  {12pt}% measure of space to leave below the theorem. E.g.: 3pt
  {}% name of font to use in the body of the theorem
  {}% measure of space to indent
  {\bfseries}% name of head font
  {:}% punctuation between head and body
  {2ex}% space after theorem head; " " = normal interword space
  {}% Manually specify head
\theoremstyle{homework} 

% Set up an Exercise environment and a Solution label.
\newtheorem*{exercisecore}{Exercise \@currentlabel}
\newenvironment{exercise}[1]
{\def\@currentlabel{#1}\exercisecore}
{\endexercisecore}

\newcommand{\localhead}[1]{\par\smallskip\noindent\textbf{#1}\nobreak\\}%
\newcommand\solution{\localhead{Solution:}}

%%%%%%%%%%%%%%%%%%%%%%%%%%%%%%%%%%%%%%%%%%%%%%%%%%%%%%%%%%%%%%%%%%%%%%%%
%
% Stuff for getting the name/document date/title across the header
\makeatletter
\RequirePackage{fancyhdr}
\pagestyle{fancy}
\fancyfoot[C]{\ifnum \value{page} > 1\relax\thepage\fi}
\fancyhead[L]{\ifx\@doclabel\@empty\else\@doclabel\fi}
\fancyhead[C]{\ifx\@docdate\@empty\else\@docdate\fi}
\fancyhead[R]{\ifx\@docauthor\@empty\else\@docauthor\fi}
\headheight 15pt

\def\doclabel#1{\gdef\@doclabel{#1}}
\doclabel{Use {\tt\textbackslash doclabel\{MY LABEL\}}.}
\def\docdate#1{\gdef\@docdate{#1}}
\docdate{Use {\tt\textbackslash docdate\{MY DATE\}}.}
\def\docauthor#1{\gdef\@docauthor{#1}}
\docauthor{Use {\tt\textbackslash docauthor\{MY NAME\}}.}
\makeatother

% Shortcuts for blackboard bold number sets (reals, integers, etc.)
\newcommand{\Reals}{\ensuremath{\mathbb R}}
\newcommand{\Nats}{\ensuremath{\mathbb N}}
\newcommand{\Ints}{\ensuremath{\mathbb Z}}
\newcommand{\Rats}{\ensuremath{\mathbb Q}}
\newcommand{\Cplx}{\ensuremath{\mathbb C}}
%% Some equivalents that some people may prefer.
\let\RR\Reals
\let\NN\Nats
\let\II\Ints
\let\CC\Cplx

%%%%%%%%%%%%%%%%%%%%%%%%%%%%%%%%%%%%%%%%%%%%%%%%%%%%%%%%%%%%%%%%%%%%%%%%%%%%%%%%%%%%%%%
%%%%%%%%%%%%%%%%%%%%%%%%%%%%%%%%%%%%%%%%%%%%%%%%%%%%%%%%%%%%%%%%%%%%%%%%%%%%%%%%%%%%%%%
% 
% The main document start here.

% The following commands set up the material that appears in the header.
\doclabel{STAT 401: Homework 5}
\docauthor{Stefano Fochesatto}
\docdate{\today}


%\begin{figure}[H]
%  \begin{center}
%  \caption{}
%  \includegraphics[\textwidth]{}
%  \end{center}
%\end{figure}

% \textbf{Code:}
% \begin{center}
% \lstinputlisting{}
% \end{center} 



\begin{document}

\begin{exercise}{1}Consider the national Football League data se named table.b1 in the MPV package. 
  Use table.b1 to learn about its contents. Then do the following:

  \begin{enumerate}
    \item[a.] Fit a MLR model relating the number of games won to the team's passing yardage($x_2$),
    the percentage of rushing plays($x_7$), and the opponents yards rushing($x_8$). Calculate t-statistics for testing the hypothesis, 
    \begin{equation*}
      H_0: \beta_2 = 0
    \end{equation*} 
    \begin{equation*}
      H_0: \beta_7 = 0
    \end{equation*} 
    \begin{equation*}
      H_0: \beta_8 = 0
    \end{equation*} 
    \solution 
     \textbf{Code:}
     \begin{center}
     \lstinputlisting{r1.txt}
     \end{center} 
     \newpage

     \item[b.] Fit a $95\%$ confidence interval on $\beta_7$ and provide an interpretation of it. \\
    \solution From the given confidence interval we know that $95\%$ of the time the true value of $\beta_7$
    will be between $(0.011855322,0.376065098)$. More specifically, $95\%$ of the time we can expect the number of games won by a team 
    to increase between $(0.011855322,0.376065098)$ for each percentage increase of rushing plays.  \\
     \textbf{Code:}
     \begin{center}
     \lstinputlisting{r2.txt}
     \end{center} 
     \newpage

     \item[c.] Find a $95\%$ confidence interval on the mean number of games won by a team when $x_2 = 2300$,
     $x_7 = 56.0$, and $x_8 = 2100$.\\
     \solution
     \textbf{Code:}
     \begin{center}
     \lstinputlisting{r3.txt}
     \end{center} 
     \newpage

     \item[d.] Find a $95\%$ prediction interval on the mean number of games won by a team when $x_2 = 2300$,
     $x_7 = 56.0$, and $x_8 = 2100$.\\
     \solution
     \textbf{Code:}
     \begin{center}
     \lstinputlisting{r4.txt}
     \end{center} 
  \end{enumerate}
\end{exercise}





\newpage
 
\begin{exercise}{2} Data on last year's sales (y, in 100,000s of dollars) in 15 sales districts are give in the file 'sales'
  posted on Canvas. this file also contains promotion expenditures ($x_1$ in the thousands of dollars), the number 
  of active accounts $(x_2)$, the number of competing brands $(x_3)$ , and the district potential $(x_4)$ for each 
  of the districts.\\
  A model with all four regressors is proposed, 
  \begin{equation*}
    y = \beta_0 + \beta_1x_1 + \beta_2x_2 + \beta_3x_3 + \beta_4x_4 + e, e ~ N(0, \sigma^2)
  \end{equation*}
  Test the following hypothesis: 
  \begin{enumerate}
    \item[a.] $\beta_4 = 0$,
    \solution 
    Fitting the model and computing the test statistic we get the following, 
    \solution
    \textbf{Code:}
    \begin{center}
    \lstinputlisting[basicstyle = \small]{r5.txt}
    \end{center} 
    With a p-value of $.537$, we fail to reject the null hypothesis and therefore at the  $\alpha = .05$ level 
    there is no statistically significant relationship between district potential and sales. Furthermore it is 
    likely that our model would attain higher parsimony by dropping the $x_4$ parameter. 
    \newpage


    \item[b.]$\beta_2 = \beta_3 = 0$\\
    \solution For this hypothesis we will need to substitute our values for $\beta_2, \beta_3$ to create a new
    simpler model and compute the F statistic. By substitution our Null model looks like, 
    \begin{equation*}
      y_{null} = \beta_0 + \beta_1x_1 + \beta_4x_4 + e, e ~ N(0, \sigma^2).
    \end{equation*}
    Fitting the model, computing the F-statistic and p-value,\\ 
    \textbf{Code:}
    \begin{center}
    \lstinputlisting[basicstyle = \small]{r6.txt}
    \end{center} 
    With a p-value of $3.957e-13$ we reject the null hypothesis and therefore at the $\alpha = .05$ the 
    alternative model achieves a greater and statistically significant amount of parsimony.
    \newpage


   \item[c.]$\beta_2 = \beta_3$\\
   \solution Again substituting $\beta_2 = \beta_3$ into our model to obtain a simplified model we get, 
   \begin{equation*}
    y_{null} = \beta_0 + \beta_1x_1 + \beta_2x_2 + \beta_2x_3+ \beta_4x_4 + e, e ~ N(0, \sigma^2).
  \end{equation*}
  \begin{equation*}
    y_{null} = \beta_0 + \beta_1x_1 + \beta_2(x_2 +x_3) + \beta_4x_4 + e, e ~ N(0, \sigma^2).
  \end{equation*}
  Fitting the model, computing the F-statistic and p-value, \\
   \textbf{Code:}
   \begin{center}
   \lstinputlisting[basicstyle = \small]{r7.txt}
   \end{center} 
   With a p-value of $4.42e-13$ we reject the null hypothesis and therefore at the $\alpha = .05$ the 
   alternative model achieves a greater and statistically significant amount of parsimony.
   \newpage

   \item[d.] $\beta_1 = \beta_2 = \beta_3 = \beta_4 = 0$\\
   \solution Recall that the omnibus test is included in the model summary, looking at the 
   summary of our full model we get, \\
   \textbf{Code:}
   \begin{center}
   \lstinputlisting[basicstyle = \small]{r8.txt}
   \end{center} 
   With a p-value of $1.285e-12$ we reject the null hypothesis and therefore at the $\alpha = .05$ the 
   alternative model achieves a greater and statistically significant amount of parsimony.
   \newpage
  \end{enumerate}  
\end{exercise}



\begin{exercise}{3.} The variable $Y$ is believed to be associated with the variables $x_1, x_2, x_3,$ and $x_4$.
  All possible subsets of these variables are used in fitting a multiple linear regression model and the $RSS$ 
  and its df of the mode are recorded below, 
    \begin{figure}[H]
    \begin{center}
    \caption{MLR models, RSS, df}
    \includegraphics[width = \textwidth]{p1.png}
    \end{center}
    \end{figure}
  \begin{enumerate}
    \item[a.] Create an ANOVA table for the full linear model using Type I sums of squares. Include $F$ statistics 
    and p-values for testing individual predictors. \\

    \solution
    Since Type I sum of squares is sequential we can can compute the sum of squares for each source with the following equation($x_0$ means 
    no variables included), 
    \begin{equation*}
      SS_{x_i} =RSS(\sum_{i = 0}^{i-1} x_i) - RSS(\sum_{i = 0}^i x_i)  
    \end{equation*}
    MS and F-statistic are computed by definition with the following, 
    \begin{equation*}
      MS = \dfrac{SS}{df}
    \end{equation*}
    \begin{equation*}
    F = \dfrac{MSR}{MSE}
    \end{equation*}
    The p-values were computed using r with the following code,  \{1 - pf(F, df of $x_1$,df of Error)\}
    
    \begin{equation*}
      \begin{bmatrix}
        Source & SS & df & MS & F & p\\
        x_1 & 3.06 & 1 & 3.06 &.71 &  0.4032303\\   
        x_2 & 453.24 & 1& 453.24 &105.40 & 3.330669e-14\\
        x_3 & 467.58 & 1& 467.58 & 108.74 & 1.909584e-14\\
        x_4 & 148.04 & 1& 148.04 & 34.43 & 2.93761e-07\\
        Error & 228.14 & 53 & 4.30 & NA&NA \\
        Total & 1300.6 & 57 & NA & NA& NA 
      \end{bmatrix}
    \end{equation*}
    \newpage

    \item[b.] Create an ANOVA table for the full linear model using Type II sums of squares. Include $F$ statistics and p-values 
    for testing individual predictors.\\
    \solution
    Type II sum of squares are computed with all other variables in the regression included. Therefore they can be computed with the following, 
    \begin{equation*}
      SS_{x_i} = RSS(x_1 + x_2 + x_3 + x_4 - x_i) - RSS(x_1 + x_2 + x_3 + x_4)
    \end{equation*}
   
    \begin{equation*}
      \begin{bmatrix}
     Source & SS & df & MS & F & p\\
        x_1 & 23.92   & 1  &  23.92  & 5.56& 0.02209803\\
        x_2 & 470.32& 1 & 470.32 & 109.38&  1.709743e-14\\
        x_3 & .05& 1 & .05 & .01& 0.9207216\\
        x_4 & 148.04& 1 & 148.04 & 34.43& 2.93761e-07\\
      Error & 228.14  & 53 & 4.30 &NA&NA\\ 
      Total & 970.47& 57 & NA &NA&NA 
    \end{bmatrix}
    \end{equation*} 
    \newpage

    \item[c.] What is $SSreg$ in both of the previous ANOVA tables, and in which table do the predictors 
    squares add up to it?\\
    \solution Firstly, Type II sums of squares do not form a perfect decomposition of SSreg. You could sum the 
    over all the sums of squares in the second ANOVA(type II) table but you wouldn't recover the SSreg for the full model. The Type I
    sums of squares do decompose SSreg, so you could sum over all the sums of squares in the first ANOVA(Type I) table to recover SSreg. 
    Doing so you get, 
    \begin{equation*}
      SSreg = 3.06 + 453.24+ 467.58 + 148.04 = 1071.92.
    \end{equation*}
    The $SSreg$ for the second ANOVA table would come out to, 
    \begin{equation*}
      23.92  + 470.32+ .05+ 148.04 = 642.33.
    \end{equation*}
    \newpage
  \end{enumerate}

  \item[d.] What is the $R^2$ coefficient in the full model?.\\
  \solution Recall the following definition of $R^2$,
  \begin{equation*}
    R^2 = \frac{SSreg}{SYY} = \frac{SSreg}{SSreg + SSR}.
  \end{equation*}
  Substituting our values for $SSreg$ and $SSR$ found in the ANOVA tables above (mainly the first one)
  we get,
  \begin{equation*}
    R^2 = \frac{SSreg}{SSreg + SSR} = \frac{1071.92}{1071.92 + 228.14} = 0.82451
  \end{equation*}
\end{exercise}



\end{document}





















