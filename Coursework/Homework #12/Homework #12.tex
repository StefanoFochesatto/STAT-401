
%%%%%%%%%%%%%%%%%%%%%%%%%%%%%%%%%%%%%%%%%%%%%%%%%%%%%%%%%%%%%%%%%%%%%%%%%%%%%%%%%%%%%%%
%%%%%%%%%%%%%%%%%%%%%%%%%%%%%%%%%%%%%%%%%%%%%%%%%%%%%%%%%%%%%%%%%%%%%%%%%%%%%%%%%%%%%%%
% 
% This top part of the document is called the 'preamble'.  Modify it with caution!
%
% The real document starts below where it says 'The main document starts here'.

\documentclass[12pt]{article}

\usepackage{amssymb,amsmath,amsthm}
\usepackage[top=1in, bottom=1in, left=1.25in, right=1.25in]{geometry}
\usepackage{fancyhdr}
\usepackage{enumerate}
\usepackage{listings}
\usepackage{graphicx}
\usepackage{float}

\usepackage{mwe}
\usepackage{caption}
\usepackage{subcaption}
% Comment the following line to use TeX's default font of Computer Modern.
\usepackage{times,txfonts}



\makeatletter
\renewcommand*\env@matrix[1][*\c@MaxMatrixCols c]{%
  \hskip -\arraycolsep
  \let\@ifnextchar\new@ifnextchar
  \array{#1}}
\makeatother

\newtheoremstyle{homework}% name of the style to be used
  {18pt}% measure of space to leave above the theorem. E.g.: 3pt
  {12pt}% measure of space to leave below the theorem. E.g.: 3pt
  {}% name of font to use in the body of the theorem
  {}% measure of space to indent
  {\bfseries}% name of head font
  {:}% punctuation between head and body
  {2ex}% space after theorem head; " " = normal interword space
  {}% Manually specify head
\theoremstyle{homework} 

% Set up an Exercise environment and a Solution label.
\newtheorem*{exercisecore}{Exercise \@currentlabel}
\newenvironment{exercise}[1]
{\def\@currentlabel{#1}\exercisecore}
{\endexercisecore}

\newcommand{\localhead}[1]{\par\smallskip\noindent\textbf{#1}\nobreak\\}%
\newcommand\solution{\localhead{Solution:}}

%%%%%%%%%%%%%%%%%%%%%%%%%%%%%%%%%%%%%%%%%%%%%%%%%%%%%%%%%%%%%%%%%%%%%%%%
%
% Stuff for getting the name/document date/title across the header
\makeatletter
\RequirePackage{fancyhdr}
\pagestyle{fancy}
\fancyfoot[C]{\ifnum \value{page} > 1\relax\thepage\fi}
\fancyhead[L]{\ifx\@doclabel\@empty\else\@doclabel\fi}
\fancyhead[C]{\ifx\@docdate\@empty\else\@docdate\fi}
\fancyhead[R]{\ifx\@docauthor\@empty\else\@docauthor\fi}
\headheight 15pt

\def\doclabel#1{\gdef\@doclabel{#1}}
\doclabel{Use {\tt\textbackslash doclabel\{MY LABEL\}}.}
\def\docdate#1{\gdef\@docdate{#1}}
\docdate{Use {\tt\textbackslash docdate\{MY DATE\}}.}
\def\docauthor#1{\gdef\@docauthor{#1}}
\docauthor{Use {\tt\textbackslash docauthor\{MY NAME\}}.}
\makeatother

% Shortcuts for blackboard bold number sets (reals, integers, etc.)
\newcommand{\Reals}{\ensuremath{\mathbb R}}
\newcommand{\Nats}{\ensuremath{\mathbb N}}
\newcommand{\Ints}{\ensuremath{\mathbb Z}}
\newcommand{\Rats}{\ensuremath{\mathbb Q}}
\newcommand{\Cplx}{\ensuremath{\mathbb C}}
%% Some equivalents that some people may prefer.
\let\RR\Reals
\let\NN\Nats
\let\II\Ints
\let\CC\Cplx

%%%%%%%%%%%%%%%%%%%%%%%%%%%%%%%%%%%%%%%%%%%%%%%%%%%%%%%%%%%%%%%%%%%%%%%%%%%%%%%%%%%%%%%
%%%%%%%%%%%%%%%%%%%%%%%%%%%%%%%%%%%%%%%%%%%%%%%%%%%%%%%%%%%%%%%%%%%%%%%%%%%%%%%%%%%%%%%
% 
% The main document start here.

% The following commands set up the material that appears in the header.
\doclabel{STAT 401: Homework 12}
\docauthor{Stefano Fochesatto}
\docdate{\today}


%\begin{figure}[H]
%  \begin{center}
%  \caption{}
%  \includegraphics[\textwidth]{}
%  \end{center}
%\end{figure}

% \textbf{Code:}
% \begin{center}
% \lstinputlisting{}
% \end{center} 



\begin{document}

\begin{exercise}{1}Use the Highway data involved in probem 10.2. Use the response $log(rate \times len)$ and treat lwid as the 
  focal regressor. Test the significance of lwid in explaining the response. Use quidelines form Section 2 in the lecture videos to d
  determine which of the other regressors (adt, trks, lane, acpt, sigs, itg, slim, shld, and htype) to test lwid in the presence of . Assume
  that scientific considerations dictate that acpt and slim be included in the model that test lwid. Interpret the results of your test. \\

  \solution Given that our model must included acpt and slim, let's first test the significance of lwid in the presence of those predictors 
  as the first guidelines in section 2 states. To do so we can simply fit the model , and the model summary will give us a significance test for 
  the lwid predictor. Doing so we get a p-value of .060234 and lwid is insignificant on the $\alpha = .05$ level. \\
  Following the section two guidelines, we now want to test lwid in the presence of moderately to low correlated predictors. Testing the predictors we get that 
  they all exhibit low correlation with lwid, so now we test the significance of lwid in the model that includes all other predictors. Doing so we get a p-value of 
  0.11658 so lwid is not a significant predictor. \\
  Since there were now high correlation predictors we would stop here and likely conclude tht lwid should not be included in the model. \\

 \textbf{Code:}
 \begin{center}
 \lstinputlisting[basicstyle = \footnotesize]{r1.txt}
 \end{center} 
\end{exercise}
\newpage

\begin{exercise}{2} Using these 'data' with a response $Y$ and three regressors $X_1, X_2$ and $X_3$ from Mantel, apply the forward 
  selection and backward elimination algorithms, using AIC as a criterion function. Also, find AIC and BIC for all possible models and compare results. Which appear 
  to be the active regressors. \\
  \solution Given that the data is very small we can just use the dredge() command from the MuMLN package we can quickly compute all possible models and their AIC and BIC (I recognize the 
  point of stepwise regression is to avoid this). Performing forward substitution we get the model $lm(Y\sim X_3)$ which just fits $X_3$ since, it has an $AIC$ which is lower than the null and the lowest 
  compared to all other single predictor models. The resulting possible models which include $X_3$ give higher AIC so we stick with $lm(Y\sim X_3)$.\\

  Backward elimination gives us that the model $lm(Y\sim X_1 + x_2)$ is the best. Note that we start with the full model, and removing $X_3$ gives us the lowest AIC of all models so we stop there and stick with 
  $lm(Y \sim X_1 + x_2)$. \\

  Note that we can also compute the BIC using the dredge() command and we still would get the same models. From our test it seems as though $X_1$ and $X_2$ are the most active regressors. \\ 

 \textbf{Code:}
 \begin{center}
 \lstinputlisting[basicstyle = \footnotesize]{r2.txt}
 \end{center} 
\end{exercise}




\newpage


\begin{exercise}{3} Use the galapagos data described in problem 10.6. Regard NS as the response and Area, Anear, Dist, Dist SC, and 
  Elevation as the possible regressors. Assume Elevation equals 80m for Baltra, 10m for Coamano, 38 m for Daphne Major, 71m for Eden, 23m for Las Plazas, and 28m for Seymour. Fit 
  a linear model with LASSO with three values of $\lambda:$ .3, .2, and .1. Report the regressors your three models admit and compare their coefficient estimates. \\
  \solution Filling the NA values and fitting teh lasso models in r we get, \\
 \textbf{Code:}
 \begin{center}
 \lstinputlisting[basicstyle = \footnotesize]{r3.txt}
 \end{center} 
  It seems as though all lasso models reported the same regressors for each $\lambda$ level. There is also very little discrepancy in the size of each 
  regressor coefficient across each model. 
\end{exercise}





\end{document}





















